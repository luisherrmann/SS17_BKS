\documentclass[numbers=noendperiod]{scrartcl}
\usepackage[utf8]{luainputenc}
\usepackage[T1]{fontenc}
\usepackage[ngerman]{babel}
\usepackage[a4paper,margin=0.75in, bottom=1in]{geometry}
\usepackage{enumerate}
\usepackage{minted}
\usepackage{mdframed}
\usepackage{courier}
\usepackage{hyperref}
\usepackage{graphicx}
\usepackage{subcaption}
\usepackage{amsmath}
\begin{document}
	
	\setlength{\parindent}{0em} 
	
	\definecolor{bg}{RGB}{230,230,230}
	\newcommand{\inputmintedframed}[2]{
		\begin{mdframed}[linecolor=bg,backgroundcolor=bg]
			\inputminted[mathescape,breaklines,linenos,numbersep=5pt,tabsize=3]{#1}{#2}
	\end{mdframed}}
	
	\hrulefill
	\begin{center}
		\bfseries % Fettdruck einschalten
		\sffamily % Serifenlose Schrift
		\begin{huge}
			Betriebs- und Kommunikationssysteme
		\end{huge}\\
		\begin{Large}
			Sommersemester 2017, 9. Übungsblatt
		\end{Large}\\
		\begin{small}
			Christoph Husemann, Luis Herrmann; Tutor: André Schröder; Mi 16:00-18:00
		\end{small}
		
		\vspace{-10pt}
	\end{center}
	\hrulefill
	
\section*{Aufgabe 1}
\subsection*{a) Aufgaben des Data-Link-Layers (Sicherungsschicht)}
Data-Link-Layers bekommen als 2-Schicht von den physical-Layers (Schicht 1) (Bit-Übertragungsschicht) einen Strom von Bits und IP-Pakete von dem Network layer(Schicht 3). Die Aufgaben des Data-Link-Layers sind insbesondere:
\begin{enumerate}
	\item Fehler erkennen und korrigieren
	\item Transparente Übertragung von Daten
	\item Rahmen (Frames)
	\begin{enumerate}
		\item Rahmen für Bitstreams zum physical-layer bilden
		\item Rahmen in Bitstreams des Network layers erkennen 
		\item Trailer mit Prüfsumme (Frame check sequence FCS) und ggfl. Header hinzufügen
		\item Bitstuffing der Nutzdaten via Hardware, um Flags für den Rahmen von Nutzdaten zu unterscheiden
	\end{enumerate}
	\item Ermitteln der best möglichen Übertragungsgeschwindigkeit zum nächsten Host (z.B. Router) (Flow control)
	\item Verbindung zum Host aufbauen/schließen
	
\end{enumerate}
\subsection{b) Angabe des Bitstrings der Ascii-Zeichen ''?~''}
Ascii Wert von ? ist: 0x3F = 0011 1111 \\
Ascii Wert von ~ ist: 0x7E = 0111 1110 \\
\(\Rightarrow \) der Bitstring der Ascii-Zeichen ''?~'' lautet ''0011 1111 0111 1110''
\subsubsection*{Ergänzung um die CRC16 Prüfsumme}
Das CRC16 IBM Prüfpolynom lauten \(x^{16}+x^{15}+x^2 + 1 \Rightarrow \) 1 1000 0000 0000 0101 \\
  \begin{align*}
&11 1111 0111 1110 0000 0000 0000 0000 \ / \ 1 1000 0000 0000 0101 =  10010010001001000101&\\
&11 0000 0000 0000 101&\\
   &1111 0111 1101 0110 0&\\
   &1100 0000 0000 0010 1&\\
     &11 0111 1101 0010 100&\\
	 &11 0000 0000 0000 101&\\
		&0111 1101 0001 1110 00&\\
		 &110 0000 0000 0001 01&\\
		   &1 1101 0001 1101 0100&\\
		   &1 1000 0000 0000 0101&\\
		      &101 0001 1100 1111 000&\\
		       &11 0000 0000 0000 101&\\
		       &10 0001 1100 1110 0110&\\
		        &1 1000 0000 0000 0101&\\
		        &0 1001 1100 1110 0001&\\
		        \end{align*}
		        \(\Rightarrow\) CRC16 Prüfsumme ist 1001 1100 1110 0001\\
		        \(\Rightarrow\) Bitstring:0011 1111 0111 1110 1001 1100 1110 0001
\subsubsection*{Bistuffing}
Immer nach fünf Einerbits hintereinander wird eine Null eingefügt
 \(\Rightarrow\) Bitstring: 0011 111\textbf{0}1 0111 11\textbf{0}10 1001 1100 1110 0001
\subsubsection*{Manchester Codierung}
\includegraphics[width=1.0\textwidth]{manchester.png}
\subsubsection*{Amplitudenmodulation}
%\includegraphics[width=1.0\textwidth]{manchester.png}
\section*{Aufgabe 2}


\end{document}
