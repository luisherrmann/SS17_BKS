\documentclass[numbers=noendperiod]{scrartcl}
\usepackage[utf8]{luainputenc}
\usepackage[T1]{fontenc}
\usepackage[ngerman]{babel}
\usepackage[a4paper,margin=0.75in, bottom=1in]{geometry}
\usepackage{enumerate}
\usepackage{minted}
\usepackage{mdframed}
\usepackage{courier}
\usepackage{hyperref}

\begin{document}
	
	\definecolor{bg}{RGB}{230,230,230}
	\newcommand{\inputmintedframed}[2]{
		\begin{mdframed}[linecolor=bg,backgroundcolor=bg]
			\inputminted[mathescape,breaklines,linenos,numbersep=5pt,tabsize=3]{#1}{#2}
	\end{mdframed}}
	
	\hrulefill
	\begin{center}
		\bfseries % Fettdruck einschalten
		\sffamily % Serifenlose Schrift
		\begin{huge}
			Betriebs- und Kommunikationssysteme
		\end{huge}\\
		\begin{Large}
			Sommersemester 2017, 3. Übungsblatt
		\end{Large}\\
		\begin{small}
			Christoph Husemann, Luis Herrmann; Tutor: André Schröder; Mi 16:00-18:00
		\end{small}
		
		\vspace{-10pt}
	\end{center}
	\hrulefill
	
\section*{Aufgabe 1}
\begin{enumerate}[1.]
	\item \begin{enumerate}[a)]
		\item \textbf{Thrashing} bezeichnet ein Problem, das bei Speicherverwaltungen typischerweise auftritt, wenn viele Prozesse gleichzeitig in den Speicher geladen werden müssen, sodass Page Segments von Prozessen häufig geswappt, d.h. aus dem Speicher entfernt werden. Dabei kann der Swap unmittelbar bevor der Prozess gebraucht wird erfolgen, sodass der Page Segment sofort wieder angelegt werden muss und unnötige Lese-/Schreibzugriffe erfolgen.
		
		\item Um virtuellen Speicher zu verwalten bietet sich die Verwendung eines \textbf{Transaction Lookaside Buffer} (TLB) an, welcher als Cache für kürzlich ermittelte physische Speicheradressen dient, da diese unter Annahme von Lokalität mit hoher Wahrscheinlichkeit unmittelbar nach dem erstem Aufruf erneut aufgerufen werden und dank Caching die virtuelle Adresse nicht jedes Mal unter Rechenaufwand neu auf eine physische Adresse abgebildet werden muss, sondern direkt in dem TLB abgerufen werden kann.
		
		\item Die \textbf{Memory Management Unit} (MMU) ist eine Hardwareeinheit des Prozessors, welche die Funktion eines Hardware-Supports für Speicherverwaltung, inbesondere für Paging und Speichervirtualisierung erfüllt und u.a. die Abbildung von virtuellen Adressen in physische Speicheradressen vornimmt.
		
		\item Eine \textbf{physikalische Adresse} bezeichnet eine Speicheradresse, welche als Bezeichner für eine auf einem physisch vorhandenen Speicher (z.B. RAM) vorhandenen Speicherstelle dient.
		
		\item \textbf{Logische/virtuelle Adressen} führen eine zusätzliche Abstraktionsebene ein, welche für den Nutzer die Speicheradressierung vereinfacht, da im virtuellen Adressraum des Speichers keine interne Fragmentierung existiert und Prozesse zusammenhängende Adressräume haben, wobei die virtuellen Adressen natürlich auf tatsächliche, physische Adressen verweisen, die Zuweisung aber für den Benutzer unsichtbar durch die Speicherverwaltung des Betriebssystems durchgeführt wird.
		
		\item Ein \textbf{Adressraum} bezeichnet eine Menge an Speicheradressen in einem Speicher, welche referenziert werden können.
	\end{enumerate}
	
	\item Von Fragmentierung spricht man, wenn der freie Speicher im Hauptspeicher nicht zusammenhängend ist. Wenn der freie Speicher innerhalb von Partitionen vorhanden ist, spricht man von interner Fragmentierung, wenn er außerhalb der Partitionen auftritt von externer Fragmenierung. 
	
	\textbf{Interne Fragmentierung} taucht dabei bei fixer Partitionierung auftaucht, d.h. wenn der Hauptspeicher in Blöcke fester Größe unterteilt wird, die gewissermaßen als 'Speicherschubladen' für einen Prozess dienen. Ist Speicher individueller Prozesse echt kleiner als die Partitionsgröße hat das zur Folge, dass im Hauptspeicher über mehrere Schubladen verstreut unverwendete, unzusammenhängende Speicherabschnitte vorhanden sind. \textbf{Externe Fragmentierung} tritt dagegen bei dynamischer Partitionierung auf, also wenn der Hauptspeicher in Blöcke dynamischer Größe unteilt wird. Dann werden Prozessen Partitionen genau der benötigten Größe erhalten können, aber unter Umständen ist der freie Speicher zwischen zwei Prozesspartitionen nicht groß genug, um den vom Prozess benötigten Speicher durchgängig, alloziieren zu können, sodass an einer anderen Stelle alloziiert werden muss.
	
	Beide Fälle laufen eine Verschwendung von Speicherplatz hinaus. Dieses Problem kann durch sogenanntes \textbf{Paging} verringert werden: Dabei wird der Speicher in hinreichend kleine Speicherblöcke fixer Größe, den sogenannten Page-Frames unterteilt und der Speicher eines Prozesses kann über mehrere Page-Frames verteilt werden, welche nicht notwendigerweise zusammenhängend sein müssen. Um zu wissen, wo sich die Page-Frames eines Prozesses befinden, verfügt jeder Prozess über eine Page-Table, in der die alloziierten Page-Frames des Prozesses vermerkt sind. Offenbar kann aber immer noch interne Fragmentierung auftreten, nämlich wenn $S \textrm{ mod } P \neq 0$, wobei $S$ der Prozesspeicher und $P$ die fixe Page-Größe ist, d.h. wenn der Prozessspeicher nicht perfekt in eine ganzzahlige Anzahl von Pages passt.
	
	\item Unter der Annahme, dass die Programme nicht als drei Subthreads des gleichen Prozesses, sondern als drei separate Prozesse gestartet werden, müsste jeder Prozess für die $main$-Funktion eine unterschiedliche Adresse zurückgeben, da jeder Prozess eine andere Prozesspartition im Hauptspeicher erhält. Da der Speicher allerdings virtualisiert ist und die Prozesse nur die virtuellen Adressen innerhalb des eigenen Adressraums sehen und nicht etwa die physischen Adressen, ist es naheliegend, dass alle drei Prozesse die gleiche Adresse für die $main$-Funktion zurückgeben.
	
\end{enumerate}



\section*{Aufgabe 2}

\inputmintedframed{c}{memmanage.h}

\inputmintedframed{c}{memoryManagement.c}

\inputmintedframed{c}{testMemManage.c}
	
\end{document}
