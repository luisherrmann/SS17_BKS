\documentclass[numbers=noendperiod]{scrartcl}
\usepackage[utf8]{luainputenc}
\usepackage[T1]{fontenc}
\usepackage[ngerman]{babel}
\usepackage[a4paper,margin=0.75in, bottom=1in]{geometry}
\usepackage{enumerate}
\usepackage{minted}
\usepackage{mdframed}
\usepackage{courier}
\usepackage{hyperref}

\begin{document}
	
\definecolor{bg}{RGB}{230,230,230}
\newcommand{\inputmintedframed}[2]{
	\begin{mdframed}[linecolor=bg,backgroundcolor=bg]
		\inputminted[mathescape,breaklines,linenos,numbersep=5pt,tabsize=3]{#1}{#2}
	\end{mdframed}}
	
\hrulefill
\begin{center}
	\bfseries % Fettdruck einschalten
	\sffamily % Serifenlose Schrift
	\begin{huge}
		Betriebs- und Kommunikationssysteme
	\end{huge}\\
	\begin{Large}
		Sommersemester 2017, 2. Übungsblatt
	\end{Large}\\
	\begin{small}
		Christopher Husemann, Luis Herrmann; Tutor: Katharina Klost; Mi 10:00-128:00          	\end{small}
	
	\vspace{-10pt}
\end{center}
\hrulefill

\section*{Aufgabe 1}
\subsection*{Begriffe}

\begin{enumerate}[1)]
	\item \textbf{Protection Rings} bezeichnen üblicherweise hiearchisch strukturierte Befugnisstufen, innerhalb derer Prozesse auf einem Computersystem ausgeführt werden. Die zugrundeliegende Idee ist, dass Prozesse aus Sicherheitsgründen nur mit genau den Befugnissen laufen sollten, die sie zur Ausführung ihrer Aufgabe benötigen \textit{principe of least authority}, da sonst die Integrität des Computersystems leicht kompromittiert werden könnte. Eine strikte, praktische Umsetzung dieses Konzepts für gängige monolitische Kernel ist jedoch schwer, weshalb man stattdessen auf Ringe zurückgreift, in denen Prozesse unterschiedlich mehr oder weniger Befugnisse haben.
	
	Aus Sicherheitsgründen laufen üblicherweise nur Prozesse des Betriebssystem-Kernels mit der obersten Befugnisstufe, welche direkten Zugriff auf die Hardware-Ressourcen ermöglicht, während Benutzerprozesse in Ringen mit niedrigerer Befugnis ausgeführt werden. In Prozessoren der x86-Familie spezifizieren 2Bit des Code-Segment-Registers (CS) des Prozessor den aktuellen Ringe, wobei prinzipiell Ringe 0 bis 3 möglich wären, heutige Betriebssysteme nutzen aber nur Ring 0 (Kernel Mode) und Ring 3 (User Mode). In der ARMv7-Architektur existieren 3 \textbf{privilege levels} PL0, PL1 und PL2, wobei PL2 das privilegierteste Level (Hypervisor-Modus) ist.
	
	\url{https://en.wikipedia.org/wiki/Protection_ring#cite_note-8}
	
	\url{http://infocenter.arm.com/help/index.jsp?topic=/com.arm.doc.dai0425/BABCDDJG.html}
	
	\item Eine Server-Client-Architektur bezeichnet ein Computersystem, in dem Prozesse in Server-Client-Beziehungen zueinander stehen. Dabei ist ein Client ein Prozess, welcher einen Dienst von einem anderen Prozess erfragt und ein Server ein Prozess, welcher Clients einen Dienst zur Verfügung stellt.
	
	\item 
		Monolithische Kernel:
		\begin{enumerate}
			\item Es gibt keine Unterscheidung zwischen User-Mode und Kernel-Mode, alle Prozesse laufen mit den gleichen Berechtigungen. Die Adressen des vom Syscall referezierten Codes werden vom Linker in das Programm eingebaut und direkt ausgeführt.
			
			\item Es gibt eine Unterscheidung zwischen Prozessen im User-Mode und solchen im Kernel-Mode. Ein Prozess $A$ im User-Mode starten durch einen Syscall eine Interrupt Service Routine (ISR) im Kernel-Mode; diese startet mit den übergebenen Parametern die gewünschte Serviceroutine, welche im Kernel-Mode läuft. Sobald diese terminiert, setzt die Ausführung der ISR fort und diese übergibt die Return-Werte der Serviceroutine an den aufrufenden Prozess $A$, welcher seine Ausführung im Usermodus fortsetzt. Es sind nur zwei Moduswechsel notwendig; ein Prozesswechsel KANN stattfinden wenn die Serviceroutine in einem separaten Prozess gestartet wird, ist aber nicht zwingend.
		\end{enumerate}
		Microkernel:
		\begin{enumerate}
			\item Ein Prozess $A$ im User-Mode führt zunächst eine Syscall durch und startet eine Routine des Microkernels im Kernel-Modus, welche einen Kontextwechsel samt Parameterübergabe zum angefragten Modul $B$ des Betriebssystems im Usermodus vermittelt. Sobald die Ausführung des Moduls $B$ terminiert, wird wieder die Routine des Microkernels im Kernelmodus ausgeführt, welche Kontextwechsel samt Return-Übergabe zurück zu Prozess $A$ im User-Mode verwaltet.
			
			\item Die Kommunikation zwischen Prozessen wird über die Syscalls SEND und RECV weiter abstrahiert. Ein Prozess $A$ im Usermode führt einen Syscall SEND durch, welche per Moduswechsel eine Routine des Microkernels startet, die an einen Prozess $B$ zu übergebende Parameter erhält und nach Beendigung die Ausführung von Routine $A$ im Usermodus fortsetzt. Das empfangende Modul des Betriebssystems startet zu einem späteren einen RECV-Syscall und startet eine Microkernel-Routine, welche die von $A$ zurückgelegten Parameter an $B$ übergibt und $B$ führt seine Aufgabe im Usermodus aus. Bei Beendigung der Aufgabe führt $B$ seinerseits einen SEND-Syscall durch, um die Return-Werte der Bearbeitung an die Kernel-Routine übermitteln zu können, sodass Prozess $A$ durch einen SEND-Syscall die Return-Werte abrufen kann.
		\end{enumerate}
	
	\url{http://www.cburch.com/csbsju/cs/350/notes/11/}
	\end{enumerate}

\subsection*{Tanenbaum-Torvalds-Debatte}

Die Tanenbaum-Torvalds-Debatte war eine Auseinandersetzung zwischen Andrew Tanenbaum und Linus Torwalds in der darum gestritten wurde, ob MINIX oder LINUX das bessere Betriebssystem sei. Zentraler Punkt des Streits war dabei die Kritik von Tanenbaum an dem Monolithic Kernel von LINUX, den er nicht als zukunftsfähig betrachtet und stattdessen einen Microkernel wie bei MINIX als überlegenes Kernelkonzept betrachtet. Der einzige Grund, weshalb sich Microkernel nicht durchgesetzt hätten sei die Tatsache, dass Monolithic Kernel (durch weniger Kontextwechsel) zu einer höheren Perfomance führten als Microkernel.

Dem hält Torvalds entgegen, dass der MINIX-Microkernel in der Theorie zwar schöner/besser sei als der monolitische Kernel von LINUX, praktisch jedoch nicht gut umgesetzt sei, zum Beispiel da es Probleme mit Multitasking gebe.


\url{http://www.oreilly.com/openbook/opensources/book/appa.html}
\subsubsection*{Eigene Meinung}
Da monoltische Kernel (fast) alle Betriebssystem-Komponenten enthalten, kann es leiccht zu Sicherheitslücken kommen, beispielsweise durch Fremdsoftware wie Geräte-Treiber. Diese werden im Kernel des Betriebssystems mit entsprechenden Systemrechten ausgeführt. Wenn die Geräte-Treiber selbst Schadsoftware oder auch nur Sicherheitslücken enthalten, entstehen große Sicherheitslücken für das gesamte System. Auf der anderen Seite ist die Performance in Systemen mit monolithischen Kernels deutlich besser. Die entscheidene Frage ist also, ob Priorität auf Performance oder Sicherheit liegen sollte. 

	Angesichts der massiven Sicherheitsprobleme im IT-Sektor und der Verbesserung der Performance durch Hardware, muss diese Frage zugunsten der Sicherheit beantwortet werden. Die wenigsten aktuellen Systeme haben wirklich ein Problem mit Performance, aber fast alle haben große Sicherheitsprobleme. Diese Situation wird unter anderem auch durch die weite Verbreitung monolithischer Kernels zulasten der Mikrokernels. Deshalb sollte die Entwicklung von Betriebssystemen mit Mikrokernel dringend wieder vorangetrieben werden.

\section*{Aufgabe 2}

\inputmintedframed{c}{trashcanl.c}

\end{document}
