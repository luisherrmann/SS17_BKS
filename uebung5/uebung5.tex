\documentclass[numbers=noendperiod]{scrartcl}
\usepackage[utf8]{luainputenc}
\usepackage[T1]{fontenc}
\usepackage[ngerman]{babel}
\usepackage[a4paper,margin=0.75in, bottom=1in]{geometry}
\usepackage{enumerate}
\usepackage{minted}
\usepackage{mdframed}
\usepackage{courier}
\usepackage{hyperref}

\begin{document}
	
	\definecolor{bg}{RGB}{230,230,230}
	\newcommand{\inputmintedframed}[2]{
		\begin{mdframed}[linecolor=bg,backgroundcolor=bg]
			\inputminted[mathescape,breaklines,linenos,numbersep=5pt,tabsize=3]{#1}{#2}
	\end{mdframed}}
	
	\hrulefill
	\begin{center}
		\bfseries % Fettdruck einschalten
		\sffamily % Serifenlose Schrift
		\begin{huge}
			Betriebs- und Kommunikationssysteme
		\end{huge}\\
		\begin{Large}
			Sommersemester 2017, 5. Übungsblatt
		\end{Large}\\
		\begin{small}
			Christoph Husemann, Luis Herrmann; Tutor: André Schröder; Mi 16:00-18:00
		\end{small}
		
		\vspace{-10pt}
	\end{center}
	\hrulefill
	
\section{Aufgabe 1}
\begin{tabular}{lcccc}
	Zeitpunkt & P1 & P2 & P3 & P4 \\
	0 & new & - & - & - \\
	1 & running & new & - & - \\
	2 & running & ready  & new & - \\
	3 & running & ready & ready & new \\
	4 & running & ready & ready & ready \\
	5 & waiting & running & ready & ready \\
	6 & waiting & running & ready & ready \\
	7 & waiting & killed & running & ready \\
	8 & waiting & killed & blocked & running \\
	9 & waiting & killed & blocked & running \\
	10 & ready & killed & ready & running \\
	11 & ready & killed & ready & running \\
	12 & ready & killed & ready & running \\
	13 & running & killed & waiting & terminated \\
	14 & running & killed & waiting & terminated \\
	15 & running & killed & waiting & terminated \\
	16 & running & killed & waiting & terminated \\
	17 & terminated & killed & waiting & terminated \\
	18 & terminated & killed & running & terminated \\
	19 & terminated & killed & running & terminated \\
	20 & terminated & killed & running & terminated \\
	21 & terminated & killed & terminated & terminated \\
	


\end{tabular}	
\section{Aufgabe 2}
	
\end{document}
